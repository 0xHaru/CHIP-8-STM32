\documentclass[a4paper]{article}

\usepackage[T1]{fontenc}
\usepackage[utf8]{inputenc}
\usepackage[english]{babel}


\usepackage[ruled,vlined,linesnumbered]{algorithm2e} %,noend


% ===== Graph =====
\usepackage{tikz}
% ===== Multicol =====
\usepackage{blindtext}
\usepackage{multicol}
% ===== Cancel =====
\usepackage{cancel}
% ===== Code =====
\usepackage{listings} 
\lstdefinestyle{mystyle}{
    breakatwhitespace=false,                   
    captionpos=b,                    
    keepspaces=true,                 
    numbers=left,                    
    numbersep=5pt,                  
    showspaces=false,                
    showstringspaces=false,
    showtabs=False,                  
    tabsize=2
}

\usepackage{hyperref}

\textwidth=450pt
\oddsidemargin=0pt
\textheight=665pt
\voffset=-50pt

\lstset{style=mystyle}

\usepackage{setspace}

\usepackage{amssymb}
\usepackage{amsthm}
\usepackage{mathtools}
\usepackage{bm}
\usepackage{setspace}

\singlespacing
% ===================
\mathtoolsset{showonlyrefs}  
\hypersetup{
    colorlinks=true,
    linkcolor=black,
    filecolor=black,      
    urlcolor=black,
}

\AtBeginDocument{\renewcommand\proofname{Proof}}

\newcommand{\pluseq}{\mathrel{{+}{=}}}

\newtheorem{theorem}{Theorem}
\newtheorem{corollary}{Corollary}
\newtheorem{lemma}{Lemma}
\newtheorem{remark}{Remark}
\newtheorem{definition}{Definition}

\setcounter{secnumdepth}{3}
\setcounter{tocdepth}{3}

% \title{Statistical Methods for Maching Learning}
% \author{Federico Bruzzone \\ Andrea Longoni \\ Luca Bellani}
% \date{}

% \makeindex
\begin{document}

% \maketitle


\begin{titlepage}
    \begin{center}
        % \includegraphics[height=2.6cm]{minerva.pdf} %{Unilogo.png}
        \includegraphics[height=5cm]{minerva.pdf} %{Unilogo.png}

        \vspace*{1.75cm}

        \LARGE
        % \text{University of Milan}

        % \vspace*{1cm}

        \textbf{MSc in Computer Science} \\
        at University of Milan

        \vspace*{1cm}

        
        \huge
        CHIP-8 ...

        \large Proposta per il Progetto di PROS, \\
               corso tenuto da \textbf{Danilo Bruschi}
        

        \normalsize
        \vspace*{4cm}

        \begin{minipage}[t]{0.47\textwidth}
	       {Email: } \vspace{0.3em} \\
              {\large \href{federico.bruzzone@studenti.unimi.it}{federico.bruzzone@studenti.unimi.it}} \vspace{1em}  \\
              {\large \href{andrea.longoni3@studenti.unimi.it}{andrea.longoni3@studenti.unimi.it}} \vspace{1em}  \\
              {\large \href{lorenzo.ferrante1@studenti.unimi.it}{lorenzo.ferrante1@studenti.unimi.it}} \vspace{1em}  \\
        \end{minipage}
        \hfill
        \begin{minipage}[t]{0.47\textwidth}\raggedleft
	       {Creato da:} \hspace{-0.9em} \vspace{0.3em} \\
              {\large \textbf{Federico Bruzzone}} \\
              \vspace{1em}
              {\large \textbf{Andrea Longoni}} \\
              \vspace{1em}
              {\large \textbf{Lorenzo Ferrante}} % \\
              % {\footnotesize mat. 123456}
        \end{minipage}

        \vfill
        Anno accademico 2022/2023
            
    \end{center}
\end{titlepage}

% \clearpage\null\newpage

% \setlength{\parskip}{0.15em}
% \tableofcontents
\setlength{\parindent}{0pt}
\setlength{\parskip}{0.8em}
\linespread{1.5}
% \setstretch{1.25}


% \clearpage\null\newpage


\section{Introduzione}

CHIP-8 é un linguaggio di programmazione interpretato, sviluppato da Joseph Weisbecker. Inizialmente è stato utilizzato sui microcomputer a 8 bit a metà degli anni '70. I programmi CHIP-8 vengono eseguiti su una macchina virtuale CHIP-8. È stato creato per consentire la programmazione più semplice dei videogiochi per questi computer. La semplicità di CHIP-8, la sua lunga storia e popolarità, hanno assicurato che gli emulatori e i programmi CHIP-8 siano ancora in fase di sviluppo.

Il CHIP-8 nel corso degli anni é stato esteso molte volte, tra le piú importanti possiamo citare, super-CHIP-8 (SCHIP-8) e la piú recente XO-CHIP.  

Lo scopo del progetto é costruire un emulatore CHIP-8 in grado di eseguire tutti i giochi e i programmi scritti nel linguaggio originale e queste estensioni. 


\section{Componenti Hardware}

Il componente principale del nostro progetto é un microcontrollore basato su architettura ARM Cortex-M4 72 \texttt{MHz} con 64-\texttt{KB} di Flash e 16-\texttt{KB} di SRAM, il  modello esatto é \texttt{STM32F334R8T6} 64 pins. Utilizziamo questa scheda perché le ROM dei giochi CHIP-8 e SCHIP-8 necessitano di una dimensione massima di 4-\texttt{KB} di SRAM. La scheda usata durante il corso basata su architettura ARM Cortex-M0 32 \texttt{MHz} \texttt{STM32L053} ha solo 8-\texttt{KB} di SRAM, quindi non sarebbe stata sufficiente per il nostro progetto, perché in SRAM dovrá girare una Virtual Machine. 

Ovviamente, necessiteremo di un display TFT LCD (thin-film-transistor liquid-crystal display) a colori retroilluminato, per visualizzare il gioco. Questo display é basato sul controller \texttt{ILI9341} e avente 2.4 pollici e ha una risoluzione di 320$\times$240 \texttt{px}. Il display dispone di un lettore di schede microSD, che utilizzeremo per caricare i giochi dal calcolatore alla scheda.

Per interagire con il gioco, utilizzeremo una tastiera matriciale 4$\times$4, in cui ogni tasto corrisponde ad un tasto della tastiera originale. La tastiera è adatta per l'uso sia con i controller da 3,3\texttt{V} che 5\texttt{V} e le dimensioni sono 69 \texttt{mm} $\times$ 77 \texttt{mm} $\times$ 1 \texttt{mm}.  

Per sentire gli effetti sonori generati dal gioco, utilizzeremo un bipper che riproduce suoni a frequenza variabile. Utilizzeremo quello fornito dal kit di sviluppo.

In aggiunta, sará possibile aggiungere uno slot per l'alimentazione via due batterie \texttt{AAA}.

\section{Componenti Software}

Le parti software che implementeremo sono:

\begin{enumerate}
    \item La \textbf{Virtual Machine} (VM) esegue i programmi scritti in CHIP-8 e SCHIP-8. Inizialmente, il testing della VM verrà effettuato su utilizzando una libreria C tra \texttt{SDL2} o \texttt{Raylib} su un calcolatore \textrm{x}86 a 64 \texttt{bit}. Successivamente, verrà portata sul microcontrollore ARM Cortex-M.
    \item Il \textbf{Driver Video} che permette l'interfacciamento tra il microcontrollore e il display TFT LCD. Lo schermo supporta interfacce parallele 6800/8080 a 8 \texttt{bit} e SPI a 3/4 fili, per ragioni di prestazioni noi implementeremo il driver utilizzando l'interfaccia parallela a 8 \texttt{bit}.  
    \item Il \textbf{Driver microSD} che permette l'interfacciamento tra il microcontrollore e il lettore della scheda microSD, verrá fatto utilizzando l'interfaccia SPI. Per questo driver utilizzeremo le librerie fornite da ST per la gestione del filesystem \texttt{FAT32} e facilitare lo sviluppo del trasferimento dei dati. 
    \item Il \textbf{Driver Keypad} che permette l'interfacciamento tra il microcontrollore e la tastiera matriciale 4$\times$4. La gestione degli input da tastiera non verrà effettuata in polling, ma bens\'{i} tramite interrupt.
    \item Il \textbf{Driver Sound} che permette l'interfacciamento tra il microcontrollore e il bipper. Il driver permette di generare suoni a frequenza variabile.
\end{enumerate}

Come ultima parte software, implementeremo un men\'{u} all'avvio dell'emulatore per la selezione del gioco da eseguire. Abbiamo deciso di non usare libereie esterne ma di implementare direttamente una libreria per fare il randering del font e scrivere sul framebuffer.  

\section{Analisi Tempi e Costi}

\begin{center}
\begin{table}[h]
    \centering
    \begin{tabular}{|llll|l|}
        \hline
        \multicolumn{1}{|l|}{\textbf{Nome}}          & \multicolumn{1}{l|}{\textbf{Modello}}         & \multicolumn{1}{l|}{\textbf{Costo unitario}}  & \textbf{Unità}  & \textbf{Costo} \\ \hline
        \multicolumn{1}{|l|}{Schermo}                & \multicolumn{1}{l|}{ILI9341 2.4"}             & \multicolumn{1}{l|}{6.44}                     & 1               & 6.44          \\ \hline
        \multicolumn{1}{|l|}{Matrix Array Keypad}    & \multicolumn{1}{l|}{AZ-Delivery 4$\times$4}   & \multicolumn{1}{l|}{3.99}                     & 1               & 3.99          \\ \hline
        \multicolumn{1}{|l|}{Microcontrollore} & \multicolumn{1}{l|}{STM32 F334R8T6}    & \multicolumn{1}{l|}{14.98}                     & 1              & 14.98          \\ \hline
        \multicolumn{1}{|l|}{Breadboard - Beeper}     & \multicolumn{1}{l|}{ELEGOO Electronic Kit}       & \multicolumn{1}{l|}{15.99}                     & 1              & 15.99          \\ \hline
        \multicolumn{4}{|r|}{\textbf{Totale}}                                                                                                            & 41.40€         \\ \hline
    \end{tabular}
    \caption{
        Materiali previsti per la costruzione del progetto. I costi indicati
        provengono da negozi online come Amazon, eBay e Aliexpress
    }
\end{table}
\end{center}

\begin{center}
\begin{table}[h]
    \centering
    \begin{tabular}{|l|l|}
        \hline
        \textbf{Nome}                         & \textbf{Tempo di lavoro}  \\ \hline
        Virtual Machine                       & 3 settimane     \\ \hline
        Driver Video                          & 1 settimana     \\ \hline
        Driver microSD                        & 3 giorni        \\ \hline
        Driver Keypad                         & 1 giorno        \\ \hline
        Driver Audio                          & 5 giorni        \\ \hline
        Men\'{u} Game Selection               & 5 giorni        \\ \hline
        Software Optimization                 & 2 settimane     \\ \hline
        \multicolumn{1}{|r|}{\textbf{Totale}} & $\min$ 2 mesi   \\ \hline
    \end{tabular}
    \caption{
        Tempi di realizzazione previsti per la realizzazione dei componenti software
    }
\end{table}

\end{center}

\end{document}
